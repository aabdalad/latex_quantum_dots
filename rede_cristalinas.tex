\par Rede cristalina é a designação dada para o conjunto ordenado de partículas que constituem um sólido. Cada cristal é constituído de células unitárias semelhantes que se repetem ao longo do material. A rede cristalina pode ser descrita pela rede de Bravais e pela rede recíproca.\cite{qm_fis2}

    \subsubsection{Rede de Bravis}

      \par Considerando-se uma rede cristalina infinita, o arranjo periódico dos átomos dessa rede pode ser descrito pela rede de Bravais, sendo que a posição de cada átomo é considerada um ponto em um espaço tridimensional (também pode ser representado na forma bidimensional). A rede de Bravais é o arranjo de todos esses pontos e a posição de cada ponto pode ser definida por um vetor \textbf{R}: 

      \begin{equation}\label{redeBravis_eq1}
        \mathbf{R} = n_{1}\cdot \mathbf{a_{1}} + n_{2}\cdot \mathbf{a_{2}} + n_{3}\cdot \mathbf{a_{3}},
      \end{equation}
onde $\mathbf{a_{1}}$, $\mathbf{a_{2}}$, $\mathbf{a_{3}}$ são vetores não coplanares, ou seja, são linearmente independentes, e são conhecidos como vetores primitivos e $n_1$, $n_2$, $n_3$ são números inteiros.\cite{qm_fis5}

      \par A rede de Bravais tem a característica de que independente do ponto escolhido para definir o vetor \textbf{R}, há sempre uma preservação da orientação, ou seja, a rede cristalina é a mesma independente de onde se observa.\cite{qm_fis5}

    \subsubsection{Rede Recíproca}

      \par Dado o conjunto de pontos \textbf{R} que constituem uma rede de Bravais e uma onda plana $e^{i \mathbf{k\cdot r}}$, tem-se que para certos vetores de onda \textbf{G}, a onda plana assume a periodicidade da rede de Bravais. Esse conjunto de vetores, dado por \textbf{G}, é conhecido como rede recíproca da rede de Bravais.  
      
      \par Como a questão da periodicidade é válida, a relação:

      \begin{equation}\label{redeReciproca_eq1}
        e^{i\mathbf{G}\cdot (\mathbf{r}+\mathbf{R})} = e^{i\mathbf{G}\cdot\mathbf{r}}
      \end{equation}
      deve ser satisfeita, em que $\mathbf{r}$ é o vetor correspondente à célula primitiva.

      Para que \eqref{redeReciproca_eq1} seja válida, tem-se que:

      \begin{align}\label{redeReciproca_eq2}
        e^{i\mathbf{G}\cdot\mathbf{R}}\ast e^{i\mathbf{G}\cdot\mathbf{r}} &= e^{i\mathbf{G}\cdot\mathbf{r}}\\
        e^{i\mathbf{G}\cdot\mathbf{R}}                                    &= 1
      \end{align}

      Ou seja, a rede recíproca pode ser caracterizada pelo conjunto de vetores k que satisfazem a relação \eqref{redeReciproca_eq2}. 
      
      A rede recíproca é gerada pela transforma de Fourier da rede de Bravais, uma vez que a transformada representa uma mudança de coordenadas de uma função periódica.\cite{qm_fis7} No caso da rede recíproca, ela representa uma mudança para o espaço do momento.\cite{qm_fis8} Como a rede recíproca é definida a partir da rede de Bravais, a segunda é conhecida como rede direta quando associada à rede recíproca.\cite{qm_fis5}
      
      Os vetores da rede recíproca estão relacionados com os vetores primitivos a partir de:
      
      \begin{align}\label{redeReciproca_eq3}
        \mathbf{b_{1}} &= 2 \Pi \frac{(\mathbf{a_{2}} \times \mathbf{a_{3}})} {\mathbf{a_{1}} \cdot (\mathbf{a_{2}} \times \mathbf{a_{3}})}\\
        \mathbf{b_{2}} &= 2 \Pi \frac{(\mathbf{a_{3}} \times \mathbf{a_{1}})} {\mathbf{a_{1}} \cdot (\mathbf{a_{2}} \times \mathbf{a_{3}})}\\
        \mathbf{b_{3}} &= 2 \Pi \frac{(\mathbf{a_{1}} \times \mathbf{a_{2}})} {\mathbf{a_{1}} \cdot (\mathbf{a_{2}} \times \mathbf{a_{3}})}
      \end{align}
      e devem satisfazer

      \begin{equation}\label{redeReciproca_eq4}
        \mathbf{b_{i}}\cdot \mathbf{a_{j}} = 2\Pi \delta_{ij},
      \end{equation}
      onde $\delta_{ij}$ representa o \textit{delta de kronecker}, que obedece a seguinte relação:

      \begin{align}\label{delta_kronecker_sistema}
        \left\{
          \begin{array}{ll}
            \displaystyle \delta_{ij} &= 0,\ se \ i\neq j\\
            \displaystyle \delta_{ij} &= 1,\ se \ i = j
          \end{array}
        \right.
      \end{align}

      \par Pode-se escrever o vetor \textbf{k} como:

      \begin{equation}\label{redeReciproca_eq5}
        \mathbf{k} = k_{1}\cdot \mathbf{b_{1}} + k_{2}\cdot \mathbf{b_{2}} + k_{3}\cdot \mathbf{b_{3}}
      \end{equation}

      \par De \eqref{redeBravis_eq1} e \eqref{redeReciproca_eq5}, tem-se:

      \begin{equation}\label{redeReciproca_eq6}
        \mathbf{k} \cdot \mathbf{R} = (k_{1} \cdot \mathbf{b_{1}} + k_{2} \cdot \mathbf{b_{2}} + k_{3} \cdot \mathbf{b_{3}})\cdot(n_{1} \cdot \mathbf{a_{1}} + n_{2} \cdot \mathbf{a_{2}} + n_{3} \cdot \mathbf{a_{3}})
      \end{equation}

      \par Aplicando a equação \eqref{redeReciproca_eq4}, obtém-se:

      \begin{equation}\label{redeReciproca_eq6}
        \mathbf{k} \cdot \mathbf{R} = 2 \Pi (k_{1} \cdot n_{1} + k_{2} \cdot n_{2} + k_{3} + n_{3})
      \end{equation}

      \par Para que $e^{i\mathbf{k} \cdot \mathbf{R}} = 1$, $\mathbf{k} \cdot \mathbf{R}$ deve ser múltiplo inteiro de $2 \Pi$. Como os coeficientes $n_{i}$ são inteiros, os coeficientes $k_{i}$ também deve ser. Então \textbf{k} será um vetor da rede recíproca sempre que puder ser escrito como a combinação linear de vetores $\mathbf{b_{i}}$ com coeficientes inteiros. Assim, a rede recíproca é uma rede de Bravais, ao considerar esses vetores como sendo os vetores primitivos. Um ponto importante é que a recíproca da rede recíproca corresponde a rede direta original.\cite{qm_fis5}

    \subsubsection{Primeira zona de Brillouin}

      A primeira zona de Brillouin representa a célula primitiva de Wigner-Seitz na rede recíproca. Ela é particularmente importante porque representa o conjunto de pontos no espaço recíproco que não cruzam o plano de Bragg. As ondas que se propagam em um potencial periódico podem ser escritas como ondas de Bloch dentro das zonas de Brillouin.\cite{qm_fis6}