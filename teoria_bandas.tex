\par A equação de Schrödinger mostra que átomos confinados possuem níveis de energia quantizados, conforme foi abordado anteriormente para o caso do elétron confinado em uma caixa. Num cristal, existem aproximadamente $10^{23}$ átomos por centímetro cúbico. Como os átomos em um material estão próximos um do outro, as funções de onda dos elétrons se superpõem, principalmente as dos elétrons da camada de valência. O princípio de exclusão de Pauli afirma que eles não podem ocupar os mesmos níveis de energia e, juntamente com a ação de um potencial, é criada uma distribuição de níveis, conhecida como banda de energia. As regiões energeticamente proibidas são chamadas de \textit{gaps} de energia\cite{qm_fis6}.