\par O confinamento quântico é um  dos fenômenos mais estudados em nanotecnologia e é causado pelo confinamento espacial dos portadores de carga em uma ou mais direções por barreiras de potencial (Miller, et al. 1984). O confinamento resulta na mudança de parâmetros referentes à estrutura eletrônica de um material. O tamanho de uma partícula interfere diretamente na sua estrutura de banda e causa mudança nas propriedades do material, como será visto adiante. (Takagahara and Takeda  1992a Wise  2000 Zhao  et  al.  2004).

\par Efeitos quânticos começam a se tornar relevantes quando a dimensão dos pontos quânticos se aproxima do raio de Bohr do éxciton do material na forma bulk (aB). O tamanho do raio de Bohr do éxciton do material bulk pode ser definido como:
