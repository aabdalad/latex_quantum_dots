% INTRODUÇÃO
\bibitem{introducao1} de Figueiredo Oliveira, André Rezende. \href{http://www.infis.ufu.br/sites/infis.ufu.br/files/Anexos/Bookpage/TCC%20F%C3%8DSICA%20DE%20MATERIAIS%202009_2%20-%20ANDRE%20REZENDE.pdf}
\textit{Caracterização Óptica de Pontos Quânticos Semicondutores de CdS em Matrizes Poliméricas}. 
\bibitem{introducao2} Ekimov, Alexey I., and Alexei A. Onushchenko. \href{https://www.researchgate.net/profile/Alexey_Onushchehko/publication/234289541_Quantum_Size_Effect_in_Three-Dimensional_Microscopic_Semiconductor_Crystals/links/0c9605305fa93c4e3d000000.pdf}\textit{Quantum size effect in three-dimensional microscopic semiconductor crystals. Jetp Lett 34.6 (1981): 345-349}.
\bibitem{introducao3} Brus, Louis E. \href{http://aip.scitation.org/doi/10.1063/1.447218}\textit{Electron-electron and electron-hole interactions in small semiconductor crystallites: The size dependence of the lowest excited electronic state. The Journal of chemical physics 80.9 (1984): 4403-4409}.
\bibitem{introducao4} Murray, CBea, David J. Norris, and Moungi G. Bawendi. \href{http://pubs.acs.org/doi/abs/10.1021/ja00072a025?journalCode=jacsat}\textit{Synthesis and characterization of nearly monodisperse CdE (E= sulfur, selenium, tellurium) semiconductor nanocrystallites. Journal of the American Chemical Society 115.19 (1993): 8706-8715}.
\bibitem{introducao5} Zhu, Jun-Jie, et al. \href{https://link.springer.com/book/10.1007/978-3-642-44910-9}\textit{Quantum dots for DNA biosensing. Vol. 165. Springer, 2013}.

% INTRODUÇÃO A MECÂNICA QUÂNTICA E FÍSICA DO ESTADO SÓLIDO
  %Redes Cristalinas
\bibitem{qm_fis1} Band, Yehuda B., and Yshai Avishai. \href{https://www.elsevier.com/books/quantum-mechanics-with-applications-to-nanotechnology-and-information-science/band/978-0-444-53786-7}\textit{Quantum mechanics with applications to nanotechnology and information science. Academic Press, 2013}.
\bibitem{qm_fis2} Oliveira, Ivan S., and Vitor LB De Jesus. \href{http://www.saraiva.com.br/introducao-a-fisica-do-estado-solido-2-ed-3527998.html}\textit{Introdução à física do estado sólido. Editora Livraria da Fisica, 2005}.
  %Equação de Schrödinger
\bibitem{qm_fis3} Eisberg, Robert Martin; Resnick, Robert - \href{https://www.livrariadafisica.com.br/detalhe_produto.aspx?id=6237}\textit{Física Quântica}
\bibitem{qm_fis4} C.W. Sherwin- \href{https://www.amazon.com/Introduction-Quantum-Mechanics-C-W-Sherwin/dp/0030068851}\textit{"Introduction to quantum mechanics" - Holt, Rinehart \& Winston of Canada Ltd (1959)}. 
\bibitem{qm_fis5} Ashcroft, Neil W., and N. David Mermin. \href{https://www.livrariadafisica.com.br/detalhe_produto.aspx?id=101883}\textit{Física do estado sólido}. Cengage Learning, 2011.
\bibitem{qm_fis6} Kittel, Charles. \href{https://www.amazon.com.br/dp/8521615051/ref=asc_df_85216150515016109?smid=A1ZZFT5FULY4LN&tag=goog0ef-20&linkCode=asn&creative=380341&creativeASIN=8521615051}\textit{Introdução à física do estado sólido}. Grupo Gen-LTC, 2006.
\bibitem{qm_fis7} Thompson, D. \href{http://aapt.scitation.org/doi/abs/10.1119/1.18243}\textit{The reciprocal lattice as the Fourier transform of the direct lattice}. American Journal of Physics 64.3 (1996): 333-334.
\bibitem{qm_fis8} Wang, Gwo-Ching, and Toh-Ming Lu. \href{https://books.google.com.br/books?hl=pt-BR&lr=&id=LEC9BAAAQBAJ&oi=fnd&pg=PR6&dq=ISBN:+978-1-4614-9286-3&ots=6N688QVTZa&sig=eKxt-la7NjrX5ZWrmHEoIZpaTwQ#v=onepage&q=ISBN%3A%20978-1-4614-9286-3&f=false}\textit{RHEED Transmission Mode and Pole Figures: Thin Film and Nanostructure Texture Analysis}. Cap. 2. Springer Science \& Business Media, 2013.
\bibitem{qm_fis9} I. Prigogine, Stuart A. Rice. \href{http://onlinelibrary.wiley.com/book/10.1002/9780470142813}\textit{Advances in Chemical Physics, Volume 57}. Wiley Interscience, 2007.
\bibitem{qm_fis10} Dagotto, Elbio. \href{http://www.springer.com/la/book/9783540432456} \textit{Nanoscale Phase Separation and Colossal Magnetoresistance: The Physics of Manganites and Related Compounds}. Springer-Verlag Berlin Heidelberg, 2003 
\bibitem{qm_fis11} Rodrigues, Rafael de Lima. \href{http://sbfisica.org.br/rbef/pdf/v19_68.pdf}\textit{Mecânica Quântica na Descrição de Schrödinger}. Revista Brasileira de Ensino de Física, vol. 19, n 1, p.68-83, 1997

% FRUSTRADO - TODO: Mudar links
 \bibitem{frustrado1} Eisberg, Robert Martin Resnick, and Cota Araiza Robert. \href{http://gen.lib.rus.ec/book/index.php?md5=80CCC290E6FFF645ADF0BA24178E4C5D}\textit{Física cuántica: átomos, moléculas, sólidos, núcleos y partículas. 1994}.
 \bibitem{frustrado2} Atkins, Peter, and Loretta Jones. \href{http://gen.lib.rus.ec/book/index.php?md5=6D32E94CECA0A9BD6FFF5F1307078071}\textit{Chemical principles: The quest for insight. Macmillan, 2007}.
 \bibitem{frustrado3} Peter, Y. U., and Manuel Cardona. \href{http://gen.lib.rus.ec/book/index.php?md5=20A8507AB491C812ED2C75D08740987A}\textit{Fundamentals of semiconductors: physics and materials properties. Springer Science \& Business Media, 2010}.
 \bibitem{frustrado5} Dagotto, Elbio. \href{http://gen.lib.rus.ec/book/index.php?md5=3C621FEBFE1EBBF8B376CED188D04A84}\textit{Nanoscale phase separation and colossal magnetoresistance: the physics of manganites and related compounds. Vol. 136}. Springer Science \& Business Media, 2013.
 \bibitem{frustrado6} de Lima Rodrigues, Rafael. \href{http://sbfisica.org.br/rbef/pdf/v19_68.pdf}\textit{Mecânica Quântica na Descrição de Schrödinger.} Revista Brasileira de Ensino de Física 19.1, 1997.

 % Potencial Periódico e Teorema de Bloch
 \bibitem{bloch1} Silva, Jusciane da Costa e. \href{www.repositorio.ufc.br/bitstream/riufc/12669/1/2008_tese_jcsilva.pdf}\textit{Confinamento Quântico em Hetero-estruturas Semicondutoras de Baixa Dimensionalidade}. Universidade Fedaral do Ceará, 2008.
 \bibitem{bloch2} H. Ibach and Hans Lüth, \href{http://www.springer.com/us/book/9783540938033}\textit{Solid-State Physics: An Introduction to Principles of Materials Science}, Springer-Verlag, 2nd Ed., 1995


% Semicondutores Bulk
\bibitem{bulk1} Averill, B. A., and P. Eldredge. \href{https://2012books.lardbucket.org/books/principles-of-general-chemistry-v1.0m/index.html}\textit{Principles of General Chemistry (v. 1.0)}. Creative Commons licensed (2012): 2991.
\bibitem{bulk2} Sattler, Klaus D. \href{https://www.amazon.com/Handbook-Nanophysics-Nanoparticles-Quantum-Dots-ebook/dp/B008I9VLAI}\textit{Handbook of Nanophysics: Nanoparticles and Quantum Dots}. CRC Press, 2016.
