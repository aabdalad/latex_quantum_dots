
\par Como visto anteriormente, no material bulk a estrutura eletrônica dos materiais é constituída por  bandas de energia  e, quando as dimensões do material são reduzidas, haverá uma alteração na estrutura eletrônica do material devido aos efeitos de confinamento. Seria interessante quantificar a relação entre as mudanças na estrutura eletrônica de um ponto quântico com o seu tamanho. Uma equação desse tipo seria muito importante para caracterização de QD’s dado que é possível estimar o tamanho de uma partícula a partir da medição de alguma de suas características eletrônicas, como espectro de emissão. Com esse objetivo, L.E. Brus, um premiado cientista americano, desenvolveu uma teoria que nos fornecesse tal fórmula.

\par Como vimos anteriormente, quando um elétron é excitado da banda de valência para a banda de condução, ele deixa para trás uma região positiva chamada de buraco. A ligação resultante desse par elétron-buraco é o éxciton e pode ser aproximada por um modelo de partícula na caixa da mecânica quântica. Para explicar o comportamento do éxciton, o modelo de Brus será deduzido. A seguinte dedução do teorema de Brus foi baseada na dedução feita por Kippeny et al (2002).

\par O modelo de Brus faz as seguintes aproximações: 

\par 1 - Considera-se um nanocristal esférico e de raio R.

\par 2 - As cargas e espaços ocupados que não são do éxciton são desconsiderados.

\par 3 - A energia potencial fora do nanocristal é infinita, ou seja, o éxciton está sempre dentro da região delimitada pelo raio do nanocristal.

\par Explora-se primeiramente o hamiltoniano de apenas uma partícula carregada (e não um éxciton) num nanocristal:

\begin{align}
	\label{confinamento_9}
	\displaystyle \mathcal{H} &= -\frac{\hbar^2}{8\pi^2m_{c}}\nabla^2_{c} + V^{\star}\\
	\displaystyle 
		V^{\star} &=
		\left\{
          \begin{array}{ll}
            \displaystyle 0,\ r \leq R\\
            \displaystyle \infty, r > R
          \end{array}
        \right.
\end{align}
onde $m_{c}$ é a massa efetiva da carga e $r$ é a distância da carga em relação ao centro do ponto quântico.

\par A solução da equação de Schrodinger para esse caso pode ser obtida como no caso da partícula na caixa. Os autovalores e autovetores modificados para coordenadas esféricas são: 

\begin{align}
	\label{confinamento_10}
      \begin{array}{ll}
        \displaystyle \Psi_n(r) = \frac{1}{r\sqrt{2\pi R}}\sin\left(\frac{n\pi r}{R}\right)\\
        \displaystyle E_{n} = \frac{\hbar^2 n^2}{8m_{c}R^2},\ n=1,2,3,...
      \end{array}
\end{align}

\par Percebe-se que conforme se aumenta o raio do nanocristal, as energias de absorção decrescem.
Porém, a criação do éxciton envolve um par elétron-buraco. Para esse caso, o Hamiltoniano fica:

\begin{equation}
	\label{confinamento_11}
	\mathcal{H} = -\frac{\hbar^2}{8\pi^2 m_{e}}\nabla^2_{e} - \frac{\hbar^2}{8\pi^2 m_{h}}\nabla^2_{h}
		+ V^{\star}(S_{e}, S_{h}),
\end{equation}
onde $S_{e}$ e $S_{h}$ são as posições do elétron e do buraco, respectivamente. $m_{e}$ é a massa efetiva do elétron e $m_{h}$ a massa efetiva do buraco.

\par A parte potencial da equação acima pode ser dividida em duas partes: a primeira parte, para $r<R$, é a atração coulombiana entre o elétron e o buraco, da seguinte maneira:

\begin{equation}
	\label{confinamento_12}
	V^{\star}_{Coul}(S_{e}, S_{h}) = -\frac{e^2}{4\pi\epsilon_{b}\epsilon_{0}|S_{e} - S_{h}|}
\end{equation}
onde $\epsilon_{b}$ é a constante dielétrica para o material bulk e $\epsilon_{0}$ é a permissividade no vácuo.

\par O segundo termo importante desse potencial é devido ao efeito de polarização. Isso acontece devido à polarização causada no cristal por uma carga. Esse efeito afeta a energia da segunda carga. O termo de polarização é dado por:

\begin{equation}
	\label{confinamento_13}
	V^{\star}_{pol}(S_{e}, S_{h}) = \frac{e^2}{2} \sum_{k=1}^{\infty} \alpha_{k}
		\frac{S_{e}^{2k} + S_{h}^{2k}}{R^{2k+1}}
\end{equation}
onde $\alpha$ é dado por:

\begin{equation}
	\label{confinamento_14}
	\alpha_{k} = \frac{(\epsilon-1)(k+1)}{4\pi \epsilon_{b} \epsilon_{0}(\epsilon k + k + 1)}; \epsilon = \frac{\epsilon_{b}}{\epsilon_{out}}
\end{equation}
onde $\epsilon_{out}$ é a constante dielétrica do meio que cerca o ponto quântico.

\par Combinando as equações \eqref{confinamento_11}, \eqref{confinamento_12}, \eqref{confinamento_13} e \eqref{confinamento_14}, o hamiltoniano total para o sistema elétron-buraco no nanocristal é:

\begin{equation}
	\label{confinamento_15}
	\mathcal{H} = - \frac{h^2}{8\pi^2 m_{e}}\nabla_{e}^2 - \frac{h^2}{8\pi^2m_{h}}\nabla_{h}^2 - \frac{e^2}{4\pi \epsilon_{b}\epsilon_{0}|S_{e} - S_{h}} + \frac{e^2}{2} \sum_{k=1}^\infty \alpha_{k} \frac{S_{e}^{2k} + S_{h}^{2k}}{R^{2k+1}}
\end{equation}

\par É claro perceber que conforme $R$ tende ao infinito, o termo de polarização tende a zero, e a equação acima se torna o hamiltoniano hidrogenoide da forma \textit{bulk}. 

\par Para que se obtenham as energias entre o elétron e o buraco no éxciton, é necessário determinar a função de onda para esse sistema considerando o hamiltoniano acima. Essa função de onda pode ser dada como o produto das funções de menor energia das cargas individuais consideradas nessa dedução. Utilizando uma aproximação de primeira ordem com uma função de onda não correlacionada para esse sistema, temos que:

\begin{equation}
	\label{confinamento_16}
	\phi_{ex}(S_{e}, S_{h}) = \Psi_{1}(S_{e})\Psi_{1}(S_{h})
\end{equation}

\par Aplicando o hamiltoniano da equação \eqref{confinamento_15} na função de onda acima e resolvendo a equação de Schrodinger resultante, é possível mostrar que os autovalores que representam a energia do éxciton são da forma:

\begin{equation}
	\label{confinamento_17}
	E_{ex} = \frac{h^2}{8R^2}\left(\frac{1}{m_{e}} + \frac{1}{m_{h}} \right) - \frac{1.8e^2}{4\pi\epsilon_{b}\epsilon_{0}R} + \frac{e^2}{R} \overline{\sum_{k=1}^{\infty} \alpha_{k} \left(\frac{S}{R} \right)^{2k}}
\end{equation}

\par Essa fórmula é conhecida como Equação de Brus. O primeiro termo da parte direita da equação representa a energia cinética e é bem relevante para pequenos valores de $R$. O segundo termo se refere à atração coulombiana entre as partículas que compõem o éxciton, e o terceiro termo é devido aos efeitos de polarização.

\par Alguns autores\cite{confinamento4}\cite{confinamento5} não consideram o terceiro termo, já que quando o ponto quântico começa a ficar muito pequeno essa equação começa a desviar da realidade por conta de ser levado também em consideração a forma do nanocristal, os efeitos da estrutura cristalina nas bandas de valência e interação quântica entre o elétron e o buraco. Porém  esses tópicos são de mecânica quântica avançada e fogem do escopo deste trabalho\footnote{E do domínio de Mecânica Quântica dos autores.}. 

\par A equação \eqref{confinamento_17} descreve importantíssimas consequências do efeito de confinamento. Primeiramente, observa-se que o gap de energia entre as bandas de valência e condução aumenta num fator $R^{-2}$ conforme o tamanho do nanocristal diminui se os dois últimos termos da equação forem desconsiderados. Além disso, em um nanocristal as bandas contínuas de energia perdem a sua característica de continuidade e se tornam níveis discretos de energia com diferentes números quânticos. Os valores de gap de energia entre as bandas são dados pela soma da energia do gap na forma bulk mais a contribuição dada pela equação \eqref{confinamento_17}. Esses são os resultados centrais deste trabalho e a partir dele serão estudadas algumas propriedades ópticas.